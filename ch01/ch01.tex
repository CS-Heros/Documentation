\chapter{Chapter 1: Introduction}
The advancement of technology in our modern era has had an important role in solving many problems that are difficult to solve by ordinary methods, as technology has opened the gates wide open for innovation and creativity to solve problems and meet consumers’ needs with ease of use and high efficiency. Despite this rapid development of technology and the emergence fields such as artificial intelligence, which was developed specifically to simulate the human mind., this is not fully exploited in the medical field other than in research, though there are quite a few problems that need exactly that.
\section{Problem Definition }
While the population is always increasing, the facilities of a country's healthcare system are struggling to keep up, especially in a developing country you can easily notice the shortage in healthcare services, equipment, and even staff.\\
As a result, some curable diseases may go unnoticed and a situation that could be easily managed might escalate to a more serious problem. \\
According to recent studies, skin and subcutaneous diseases are the fourth leading cause of nonfatal disease burden globally, affecting 30–70\% of individuals and prevalent in all age groups. In addition to that Diagnostic accuracy of non-specialists is only 24–70\%.
\section{Project Objectives}
Our team took notice of that and we started thinking how we could integrate computer science and machine learning with the medical field to help mitigate the load on the healthcare system. \\
Researchers have been trying to leverage the benefits of AI technology for a few decades now and since we are passionate about solving problems, especially ones that are directly affecting human lives, we are trying to address this problem by providing a solution that can help both the patient and the professional healthcare provider.
\begin{center}
    \large\textbf{The following list shows the main objectives of the project:}
\end{center}
\begin{itemize}
  \item To use Artificial Intelligence tools as a promising method of diagnosing commonly known diseases and to be a relatively easier and a cheaper added asset to help with the lack of healthcare services and staff.\\
  \item To develop a system that can diagnose skin diseases with high accuracy using AI models that are deployed to the cloud.\\
  \item To make use of the abundance of data to be provided by users with their permission to increase the AI models accuracy.\\
  \item To develop an easy-to-use mobile application connected to our online system to be the interface of our system, which will in turn be the entry point of metadata and images from the users to be scanned.\\
  \item To provide a follow-up with the users with information and possibly off-the-shelf medicine to help with their diagnosis.
\end{itemize}
\section{Project Scope }
 “DermaAI” is an application that can provide reliable diagnosis of 620+ skin 
diseases. We believe this will help mitigate the load on the healthcare system
and provide the public with an alternative when there are no options left.\\\\
\textbf{The process of diagnosis goes as follows:}\\
First the patient requests a new scan which is followed by a couple of simple 
questions, like the affected place on the body for example.
Secondly, he is to be asked to provide an image of the area related to the 
diagnosis, he may also be asked to crop out only that particular area to make it easier to detect.\\\\
Finally, the patient is asked to confirm his input. The result will be the 
highest predicted diseases, in addition to photos of similar cases and 
information about the disease and what the patient can and can’t do to take care of the affected area.\\\\
\section{Project Timeline}
\begin{itemize}
    \item\textbf{Identifying project objectives} 
       \begin{itemize}
         \item Diagnose skin diseases with high accuracy using Deep Learning.\\
         \item Ease of providing drugs related to dermatology through the App.\\
         \item Ease of booking an appointment with a specialist for the patient's condition.\\
       \end{itemize}
    \item \textbf{Project implementation is divided into 4 phases}
       \begin{itemize}
           \item First Phase (15 Oct - 7 Jan)
             \begin{itemize}
                 \item Brainstorm for Requirement Analysis 
                    \begin{itemize}
                        \item 1-Sprint
                        \item 15th Oct  To 1st Nov
                    \end{itemize}
                \item Design Document \&  Prototype  
                     \begin{itemize}
                        \item 2-Sprints
                        \item 1st Nov To 1st Dec - 2 Sprints
                    \end{itemize}
                \item Development Demo with first Layer of models  
                     \begin{itemize}
                        \item 1-Sprint
                        \item 1 Dec To 15 Dec
                    \end{itemize}
                \item Quality Assurance \& 1st version Deployment
                    \begin{itemize}
                        \item 1-Sprint
                        \item 15 Dec To 30 Dec
                    \end{itemize}
                \item Risk Management 7 days.    
              \end{itemize}
            \item Second Phase (7 Jan - 30 Feb) 
               \begin{itemize}
                  \item Brainstorm, Design Document 
                    \begin{itemize}
                        \item 1-Sprint
                        \item 7 Jan To 23 Jan
                    \end{itemize}
                 \item Development of 8 DL models in the second layer
                    \begin{itemize}
                        \item 1-Sprint
                        \item 23 Jan To 7 Feb
                    \end{itemize}
                 \item Quality Assurance for test demo for live data \& Deployment for second version 
                    \begin{itemize}
                        \item 1-Sprint
                        \item 7 Feb to 23 Feb.
                    \end{itemize}
                 \item Risk Management 7 days.
                \end{itemize}
            \item Third Phase (30 Feb - 23 Apr)
               \begin{itemize}
                   \item Brainstorm, Design Document
                      \begin{itemize}
                          \item 1-Sprint
                        \item 30 Feb to 14 Mar
                      \end{itemize}
                    \item Development Demo with of 16 DL models in the second layer
                       \begin{itemize}
                          \item 1-Sprint
                          \item 14 Mar to 30 Mar
                      \end{itemize}
                    \item Quality Assurance for test demo for live data \& Deployment for third version
                       \begin{itemize}
                          \item 1-Sprint
                          \item 1 Apr to 15 Apr.
                       \end{itemize}
                    \item Risk Management 7 days.
                \end{itemize}
            \item Fourth Phase (23 Apr - 30 Jun)
               \begin{itemize}
                   \item Brainstorm, Design Document
                      \begin{itemize}
                          \item 1-Sprint
                        \item 23 Apr to 7 May
                      \end{itemize}
                    \item Development Demo with of 23 DL models in the second layer
                       \begin{itemize}
                          \item 1-Sprint
                          \item 7 May to 23 May
                      \end{itemize}
                    \item Quality Assurance for test demo for live data \& Deployment for  final version
                       \begin{itemize}
                          \item 1-Sprint
                          \item 23 May to 7 Jun.
                       \end{itemize}
                    \item Completing Full Documentation
                        \begin{itemize}
                          \item 1-Sprint
                          \item 7 Jun to 23 Jun
                       \end{itemize}
                    \item Risk Management 7 days.
                \end{itemize}
           \item Addition Risk Management on overall plan 15 days 
       \end{itemize}
     \end{itemize}
\section{Document Organization }
This project consists of six chapters in addition to one appendix. These chapters are organized to reflect the scientific steps toward our main objective. A brief description about the contents of each chapter is given in the following paragraphs:
\subsection{Chapter 1 :}, Introduction, introduces the project objectives, the motivation of the project, the approach used in this project, and the scope of the project.
\subsection{Chapter 2 :}This chapter provides coverage of common techniques and techniques used in This project, an overview of related works related to our app and what's new in Our application, then it will show the advantages and disadvantages of these applications and Through which we inspired the idea of our application in order to make it more These features solve most of these shortcomings. Also provide a list of the libraries and tools that We used to build our app.
\subsection{Chapter 3 :}This chapter will introduce the project analysis process which includes an observation of the functions to be used in the project and shows detailed explanation about our application. Also shows some figures such as use case diagram, sequence diagram and system architecture which is a generic discipline to handle systems. 
\subsection{Chapter 4 :} This chapter, we will be keen on showing some detailed ERD for our application,class diagram, and some detailed screens on the application, with an explanation of the importance of each screen and the technology used to build it. 
\subsection{Chapter 5 : }This chapter shows the implementation of the system, shows the process of mapping design into implementation, as well as the chapter will discuss test/achieved results.
\subsection{Chapter 6 : }Conclusion and future work, summarizes the entire research, and addresses the suggested improvements for the system. 